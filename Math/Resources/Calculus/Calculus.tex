\documentclass[12pt]{article}
\usepackage{amsmath,amsthm,amssymb}
\begin{document}
\tableofcontents
\newpage
\section{Functions}
25 May 2023
\subsection{Functions. Domain of Definition}
The independent variable $x$ is defined by a set $X$ of its values.
If to each value of the independent variable $x \in X$ there corresponds
one definite value of another variable $y$, then $y$ is called
the function of $x$ with a domain of definition (or domain) $X$ or,
in functional notation, $y=y(x)$, or $y=f(x)$, or $y=\varphi(x)$, and so
forth. The set of values of the function $y(x)$ is called the range
of the given function.

\subsection{Investigation of Functions}
A function $f(x)$ defined on the set $X$ is said to be non-decreasing
on this set (respectively, increasing, non-increasing, decreasing), if
for any numbers $x_1,x_2 \in X, x_1<x_2$ the inequality $f(x_1)\le f(x_2)$ (respectively, $f(x_1)<f(x_2),f(x_1)\ge f(x_2),f(x_1)>f(x_2)$ )  is satisfied.The function $f (x)$ is said to be monotonic on the set $X$ if it possesses one of the four indicated properties. The function $f(x)$ is said to be bounded above (or below) on the set $X$ if there exists a number $M(or\thickspace m)$ such that $f(x)\le M \thickspace \forall \thickspace x \in X$.  The function $f(x)$ is said to be bounded on the set $X$ if it is bounded above and below.
The function $f(x)$ is called periodic if there exists a number
$T > 0$ such that $f(x + T)=f(x)$ for all $x$ belonging to the domain
of definition of the function (together with any point $x$ the point
$x+T$ must belong to the domain of definition). The least number $T$
possessing this property (if such a number exists) is called the
period of the function $f(x)$. The function $f(x)$ takes on the maximum value at the point $x_o\in X$ if $f(x_o)\ge f(x)$ for all $x\in X$, and the minimum value if if $f(x_o) \le f(x)$ for all $x \in X$. A function $f(x)$ defined on a set $X$ which is symmetric $w.r.t$ origin of coordinates is called even if $f(-x)=f(x)$, and odd if $f(x)=-f(x)$.

\subsection{Inverse of Function}
Let the function $y=f(x)$ be defined on the set $X$ and have a
range $Y$. If for each $y \in Y$ there exists a single value of $x$ such that $f(x) = y$, then this correspondence defines a certain function $x = g (y)$ called inverse $w.r.t$ given function $y= f(x)$. The sufficient condition for the existence of an inverse function is a strict monotony of the original function $y = f (x)$. If the function increases(decreases), then the inverse function also increases (decreases). The graph of the inverse function $x = g (y)$ coincides with that of the function $y=f(x)$ if the independent variable is marked off along the $y-axis$. If the independent variable is laid off along the $x$-axis,i. e. if the inverse function is written in the form $y = g (x)$, then the graph of the inverse function will be symmetric to that of the function $y = f (x)$ with respect to the bisector of the first and third quadrants.

\section{Limits}
\subsection{Existence}
Limit of function $f(x)$ is said to exist as $x \to a$ when,
$$\lim_{h \to 0^+} f(a-h)=\lim_{h \to 0^+} f(a+h)$$ equal to some finite value $L$.
\subsection{Indeterminate forms}
There are only seven indeterminate forms \\
$\frac{0}{0},\frac{\infty}{\infty},0 \times \infty, \infty - \infty, \infty^0, 0^0 \thickspace and \thickspace 1^{\infty}.$
\subsection{List of limits}
\textbf{Limits Operations} \\
If $\lim_{x \to c} f(x)=L$
\begin{itemize}
\item $\lim_{x \to c} [f(x \pm a)]=L \pm a$
\item $\lim_{x \to c} af(x)=aL$
\item $\lim_{x \to c} \frac{1}{f(a)}= \frac{1}{L}$ for $L>0$
\item $\lim_{x \to c} f(x)^n=L^n$ for $n>0$
\end{itemize}
\textbf{Involving infinitesimal changes} \\
If infinitesimal change $h$ if denote by $\Delta x$. If $f(x)$ and $g(x)$ are differentiable at $x$.
\begin{itemize}
\item $\lim_{h \to 0} \frac{f(x+h)-f(x)}{h}=f'(x)$
\item $\lim_{h \to 0} \frac{fog(x+h)-fog(x)}{h}=f'[g(x)]g'(x)$
\item $\lim_{h \to 0} \frac{f(x+h)g(x+h)-f(x)g(x)}{h}=f'(x)g(x)+f(x)g'(x)$
\item $\lim_{h \to 0}( \frac{f(x+h)}{f(x)} )^{\frac{1}{h}}=exp(\frac{f'(x)}{f(x)})$
\item $\lim_{h \to 0}( \frac{f(e^h x)}{f(x)} )^{\frac{1}{h}}=exp(\frac{xf'(x)}{f(x)})$
\end{itemize}
If $f(x)$ and $g(x) $are differentiable on an open interval containing $c$, except possibly $c$ itself, and $\lim_{x \to c} f(x)=\lim_{x \to c} g(x)=0$ or $\pm \infty$. \\ 
Jean Bernoulli or L'Hopital's rule can be used: $$\lim_{x \to c} \frac{f(x)}{g(x)}= \frac{f'(x)}{g'(x)}$$
\textbf{Inequalities} \\
If $f(x) \le g(x)$ for all $x$ in interval that contains $c$, except possibly $c$ itself, and the limit of $f(x)$ and $g(x)$ both exist at $c$, then $\lim_{x \to c} f(x) \le \lim_{x \to c} g(x) $ \\
If $\lim_{x \to c} f(x)= \lim_{x \to c} h(x)=L $ and $$f(x) \le g(x) \le h(x)$$ for all x in an open interval that contains $c$, except 
possibly c itself, $\lim_{x \to c} g(x)=L$. This is know as $Squeeze \thickspace Theorem.$
\subsection{Exponential Functions}
\textbf{Function of form $f(x)^{g(x)}$}
\begin{itemize}
\item $\lim_{x \to + \infty} (\frac{x}{x+k})^x=e^{-k}$
\item $\lim_{x \to 0} (1+x)^\frac{1}{x}=e$
\item $\lim_{x \to 0} (1+kx)^\frac{m}{x}=e^{mk}$
\item $\lim_{x \to + \infty} (1+ \frac{1}{x})^x=e$
\item $\lim_{x \to + \infty} (1- \frac{1}{x})^x=\frac{1}{e}$
\item $\lim_{x \to + \infty} (1+ \frac{k}{x})^{mx}=e^{mk}$
\item $\lim_{x \to 0} (1+ a(e^{-x}-1)^{- \frac{1}{x}})=e^a$
\end{itemize}
\textbf{Sum products and Composites}
\begin{itemize}
\item $\lim_{x \to 0} (\frac{a^x -1}{x})=lna$
\item $\lim_{x \to 0} (\frac{e^x -1}{x})=1$
\item $\lim_{x \to 0} (\frac{e^{ax}-1}{x})=a$
\end{itemize}
\subsection{Logarithmic Functions}
\begin{itemize}
\item $\lim_{x \to 1} \frac{lnx}{x-1}=1$
\item $\lim_{x \to 0} \frac{ln(x+1)}{x}=1$
\item $\lim_{x \to 0} \frac{-ln(1+a(e^{-x}-1)}{x}=a$
\end{itemize}
\textbf{Some cases}
\begin{itemize}
\item $\lim_{x \to 0^+} log_b x=-F(b) \infty$
\item $\lim_{x \to \infty} log_b x=F(b) \infty$
\end{itemize}
where $F(x)=2H(x-1)-1$ and $H(x)$ is Oliver Heaviside step function.
\subsection{Trigonometric Functions}
\begin{itemize}
\item $\lim_{x \to 0} \frac{ \sin ax}{ax}=1 $ for $a \not= 0$
\item $\lim_{x \to 0} \frac{\sin ax}{bx}=\frac{a}{b} $ for $b \not= 0$
\item $\lim_{x \to \infty} x \sin( \frac{1}{x})=1$
\item $\lim_{x \to 0} \frac{\tan ax}{ax}=1 $ for $a \not= 0$ 
\item $\lim_{x \to 0} \frac{\tan ax}{bx}=\frac{a}{b} $ for $b \not= 0$
\end{itemize}
\subsection{Sums}
\begin{itemize}
\item $\lim_{x \to \infty} \sum_{k=1}^{n} \frac{1}{k}= 
\infty $
\item $\lim_{x \to \infty} (\sum_{k=1}^{n} \frac{1}{k}-logn)= 
\gamma $.This is Euler Mascheroni Constant.
\end{itemize}
\subsection{Notable Special Limits}
\begin{itemize}
\item $\lim_{x \to \infty} \frac{n}{\sqrt[n]{n!}}=e$
\item $\lim_{x \to \infty} 2^n \sqrt{2- \sqrt{2 + \sqrt{2 +...+ \sqrt{2}}}}= \pi $
\end{itemize}
\subsection{Taylor Series}
$$e^x = 1 + \frac{x}{1!}+ \frac{x^{2}}{2!}+ \frac{x^{3}}{3!}+...+ \infty$$
$$ln(1+x) = x - \frac{x^{2}}{2}+ \frac{x^{3}}{3}- \frac{x^{4}}{4}+...+ \infty$$
$$ln(1-x) = -(x + \frac{x^{2}}{2}+ \frac{x^{3}}{3}+ \frac{x^{4}}{4}+...+ \infty)$$
$$ln(\frac{1+x}{1-x}) = 2(x+ \frac{x^{3}}{3}+ \frac{x^{5}}{5}+ \frac{x^{7}}{7}+...)$$
$$\sin x = x - \frac{x^{3}}{3!}+ \frac{x^{5}}{5!}- \frac{x^{7}}{7!}+...+ \infty$$
$$\cos x = 1 - \frac{x^{2}}{2!}+ \frac{x^{4}}{4!}- \frac{x^{6}}{6!}+...+ \infty$$
$$\tan x = x + \frac{x^{3}}{3}+ \frac{2x^{5}}{15}+...+ \infty$$
$$\sec x = x + \frac{x^{2}}{2}+ \frac{5x^{4}}{24}+...+ \infty$$
$$arc \sin x/ \sin^{-1}x = x + \frac{x^{3}}{6}+ \frac{3x^{5}}{40}+...+ \infty$$
$$arc \cos x/ \cos^{-1}x = \frac{\pi}{2}- (x + \frac{x^{3}}{6}+ \frac{3x^{5}}{40}+...)$$
$$arc \tan x/\tan^{-1}x = x - \frac{x^{3}}{3}+ \frac{x^{5}}{5}+...+ \infty$$
\section{Differentiation}
26 May 2023
\subsection{Elementary functions}
\begin{itemize}
\item $\frac{d}{dx} (x^n)=nx^{n-1}$
\item $\frac{d}{dx} (a^x)=a^x lna$
\item $\frac{d}{dx} (lnx)=\frac{1}{x}$
\item $\frac{d}{dx} (\log_a x)=\frac{1}{xlna}$
\item $\frac{d}{dx} (\sin x)=\cos x$
\item $\frac{d}{dx} (\cos x)=-\sin x$
\item $\frac{d}{dx} (\sec x)=\sec x \tan x$
\item $\frac{d}{dx} (\csc x)=-\csc x \cot x$
\item $\frac{d}{dx} (\tan x)=\sec^2x$
\item $\frac{d}{dx} (\cot x)=-\csc^2x$
\end{itemize}
\subsection{Basic Theorems}
\begin{itemize}
\item $\frac{d}{dx} (f \pm g)=f'(x) \pm g'(x)$
\item $\frac{d}{dx} (kf(x)=k\frac{d}{dx}(f(x))$
\item $\frac{d}{dx} (f(x).g(x))=f(x)g'(x)+g(x)f'(x)$
\item $\frac{d}{dx} (\frac{f(x)}{g(x)})=\frac{g(x)f'(x)-g'(x)f(x)}{g^2(x)}$
\item $\frac{d}{dx} (f(g(x)))=f'(g(x))g'(x)$
\end{itemize}
\subsection{Inverse Trigonometric Functions}
\begin{itemize}
\item $\frac{d}{dx} (\sin^{-1}x)=\frac{1}{\sqrt{1-x^2}}$
\item $\frac{d}{dx} (\cos^{-1}x)=\frac{-1}{\sqrt{1-x^2}}$
\item $\frac{d}{dx} (\tan^{-1}x)=\frac{1}{1+x^2}$
\item $\frac{d}{dx} (\cot^{-1}x)=\frac{-1}{1+x^2}$
\item $\frac{d}{dx} (\sec^{-1}x)=\frac{1}{|x| \sqrt{x^2-1}}$
\item $\frac{d}{dx} (\csc^{-1}x)=\frac{-1}{|x| \sqrt{x^2-1}}$
\end{itemize}
\subsection{Using Substitution}
\begin{itemize}
\item  $\sqrt{x^2+a^2} \implies x=atan \theta$
\item  $\sqrt{a^2-x^2} \implies x=asin \theta$
\item  $\sqrt{x^2-a^2} \implies x=asec \theta$
\item  $\sqrt{\frac{x+a}{a-x}} \implies x=acos \theta$
\end{itemize}
\subsection{Parametric Differentiation}
If $y=f(\theta)$ and $x=g(\theta)$ where $\theta$ is parameter then
$$\frac{dy}{dx} = \frac{\frac{dy}{d\theta}}{\frac{dx}{d\theta}}$$
\subsection{Derivative of Determinant}
If $F(x)= 
\begin{vmatrix}
f(x) & g(x) & h(x)\\
l(x) & m(x) & n(x)\\
u(x) & v(x) & w(x)
\end{vmatrix}$
where $f,g,h,l,m,n,u,v,w$ are differentiable functions, then
$F'(x)= \begin{vmatrix}
f'(x) & g'(x) & h'(x)\\
l(x) & m(x) & n(x)\\
u(x) & v(x) & w(x)
\end{vmatrix} +
\begin{vmatrix}
f(x) & g(x) & h(x)\\
l'(x) & m'(x) & n'(x)\\
u(x) & v(x) & w(x)
\end{vmatrix} +
\begin{vmatrix}
f(x) & g(x) & h(x)\\
l(x) & m(x) & n(x)\\
u'(x) & v'(x) & w'(x)
\end{vmatrix}
$
\section{Application of Derivatives}
\subsection{Equation of Tangent and Normal}
Tangent at $(x_1,y_1)$ is given by $$(y-y_1)=f'(x_1)(x-x_1)$$ when $f'(x_1)$ is real and Normal at $(x_1,y_1)$ is given by $$(y-y_1)=\frac{-1}{f'(x_1)}(x-x_1)$$ when $f'(x_1)$ is non-zero and real.\\
\textbf{Tangent from an external point} \\
Given a point $\delta(a,b)$ which does not lie on the curve $y=f(x)$
then equation of possible tangents to the curve $y=f(x)$ passing through $(a,b)$ can be found by solving for point of contact $\lambda$
$$f'(h)=\frac{f(h)-b}{h-a}$$ 
and equation of tangent $$ y-b=\frac{f(h)-b}{h-a} (x-a) $$ \\
\textbf{Length of tangent,normal,sub-tangent,sub-normal} from point $\sigma(h,k)$ and slope $m$\\
Length of Tangent $=|k|\sqrt{1+ \frac{1}{m^2}}$ \\
Length of Normal  $=|k|\sqrt{1+ m^2}$ \\
Length of Sub-Tangent $=|\frac{k}{m}|$ \\
Length of Sub-Normal $=|km|$ \\
\textbf{Angle between the curves}
$$\tan \theta=|\frac{m_1-m_2}{1+m_1m_2}|$$
\subsection{Theorems}
\textbf{Rolle's Theorem}\\
If a function $f$ defined on $[a,b]$ and
\begin{itemize} 
\item  continuous on $[a,b]$
\item derivable on $(a,b)$ 
\item $f(a)=f(b)$
\end{itemize}
then there exists at least one real number c between a and b $(a<c<b)$
such that $f'(c)=0.$ \\
\textbf{Lagrange's Mean Value Theorem } \\
If a function $f$ defined on $[a,b]$ and
\begin{itemize} 
\item  continuous on $[a,b]$
\item derivable on $(a,b)$
\end{itemize}
then there exists at least one real number c between a and b $(a<c<b)$
such that $$f'(c)=\frac{f(b)-f(a)}{b-a} $$
\textbf{Cauchy's Mean Value theorem}\\
If functions $f$  and $g$ defined on $[a,b]$ and
\begin{itemize} 
\item  continuous on $[a,b]$
\item derivable on $(a,b)$
\item $c \in (a,b)$ then
$$\frac{f'(c)}{g'(c)}=\frac{f(b)-f(a)}{g(b)-g(a)}$$
\end{itemize}
\subsection{Maxima and Minima}
If a function $y=f(x)$ is defined on interval $X$, then an interior point $x_o$ of interval is called the point of $maximum$ of function $f(x)$ [the point of $minimum$ of function $f(x)$] if there exists a neighbourhood $U \in X$ of point $x_o$ such that inequality $f(x) \le f(x_o)[f(x) \ge f(x_o)]$ holds true within it.\\
\textbf{A Necessary condition for the existence of an  Extremum}\\ At points of extremum the derivative $f'(x)$ is equal to zero or does not exist. The points at which $f'(x)=0$ or does not exit are called $critical \thickspace points.$\\
\textbf{Sufficient conditions for the existence of an Extremum} \\
1. Let the function $f(x)$ be continuous in some neighbourhood of point $x_o$
\begin{itemize}
\item If $f'(x)>0$ at $x<x_o$ and $f'(x)<0$ at $x>x_o$ (i.e if in moving from left to right through point $x_o$ the derivative changes
sign from plus to minus), then at point $x_o$ the function reaches $maximum.$
\item If $f'(x)<0$ at $x<x_o$ and $f'(x)>0$ at $x>x_o$ (i.e if in moving through the point $x_o$ from left to right the derivative changes sign from minus to plus ), then at point $x_o$ the function reaches $minimum.$
\item If the derivative does not change sign in moving through the point $x_o$, then there is no $extremum.$
\end{itemize}
2. Let the function $f(x)$ be twice differentiable (that is $f'(x_o)=0$) at a critical point $x_o$. If $f''(x_o)<0$ then at $x_o$ the function has a $maximum$; if $f'(x_o)>0$ then at $x_o$ the function has $minimum$ but if $f''(x_o)=0$ then the question of existence of $extremum$ at this point remains open. \\
3. Let $f(x_o)=f''(x_o)=...=f^{n-1}(x_o)=0$, but $f^n(x_o) \not= 0$. If $n$ is even, then at $f^n(x_o)<0$ there is a $maximum$ at $x_o$, and at point $f^n(x_o)>0$, a $minimum$.If $n$ is odd then there is no $extremum$ at point $x_o$. \\
4.Let the function $y=f(x)$ be represented parametrically:
$$x=\varphi(t), \thickspace y=\psi(t)$$ where the functions $\varphi(t)$ and $\psi(t)$ have derivatives both of first and second orders within a certain interval of change of argument $t$, and $\varphi'(t) \not=0$. Further, let, at $t=t_o$
$$\psi'(t)=0$$
Then:
\begin{itemize}
\item If $\psi''(t_o)<0,$ the function $y=f(x)$ has a $maximum$ at $x=x_o=\varphi(t_o)$
\item If $\psi''(t_o)>0,$ the function $y=f(x)$ has a $minimum$ at $x=x_o=\varphi(t_o)$
\item If $\psi''(t_o)=0,$ the question of existence of $extremum$ remains open.
\end{itemize}
\section{Integration}
27 May 2023
\subsection{Standard Formula}
\begin{itemize}
\item $\int(ax+b)^n dx=\frac{(ax+b)^{n+1}}{a(n+1)}+c$
\item $\int \frac{dx}{ax+b}=\frac{1}{a} ln(ax+b)  +c$
\item $\int e^{ax+b}dx= \frac{1}{a} e^{ax+b}+c$
\item $\int a^{px+q}dx=\frac{1}{p}\frac{a^{px+q}}{lna} +c$
\item $\int \sin(ax+b)dx=\frac{-1}{a}\cos(ax+b)+c$
\item $\int \cos(ax+b)dx=\frac{1}{a}\sin(ax+b)+c$
\item $\int \tan(ax+b)dx=\frac{1}{a}ln \sec(ax+b)+c$
\item $\int \cot(ax+b)dx=\frac{1}{a}ln \sin(ax+b)+c$
\item $\int \sec^2(ax+b)dx=\frac{1}{a}\tan(ax+b)+c$
\item $\int \csc^2(ax+b)dx=\frac{-1}{a} \cot(ax+b)+c$
\item $\int \sec xdx=ln(\sec x+\tan x)+c$
\item $\int \csc xdx=ln(\csc x-\cot x)+c$
\item $\int \frac{dx}{\sqrt{a^2-x^2}}=\sin^{-1}(\frac{x}{a})+c$
\item $\int \frac{dx}{a^2+x^2}=\frac{1}{a}\tan^{-1}(\frac{x}{a})+c$
\item $\int \frac{dx}{|x|\sqrt{x^2-a^2}}=\frac{1}{a}\sec^{-1}(\frac{x}{a})+c$
\item $\int \frac{dx}{\sqrt{x^2+a^2}}=ln[{x+\sqrt{x^2+a^2}}]+c$
\item $\int \frac{dx}{a^2-x^2}=\frac{1}{2a}ln|\frac{a+x}{a-x}|+c$
\item $\int \frac{dx}{x^2-a^2}=\frac{1}{2a}ln|\frac{x-a}{x+a}|+c$
\item $\int \sqrt{a^2-x^2}dx=\frac{x}{2}\sqrt{a^2-x^2}+\frac{a^2}{2}sin^{-1}(\frac{x}{a})+c$
\item $\int \sqrt{x^2+a^2}dx=\frac{x}{2}\sqrt{x^2+a^2}+\frac{a^2}{2}ln(\frac{x+\sqrt{x^2+a^2}}{a})+c$
\item $\int \sqrt{x^2-a^2}dx=\frac{x}{2}\sqrt{x^2-a^2}-\frac{a^2}{2}ln(\frac{x+\sqrt{x^2-a^2}}{a})+c$
\end{itemize}
\subsection{Integration of types}
\textbf{By Partial Fraction} \\
A method of integrating rational functions that are fractions in which the denominator has a higher degree than the numerator. For example
the integral $$\int \frac{x+3}{x^2+3x+2}dx$$
can be put in the form $$\frac{A}{x+2} +\frac{B}{x+1}$$
$A$ and $B$ can be found by putting this expression in the form $$\frac{A(x+1)+B(x+2)}{x^2+3x+2}$$ Then $$x+3=(A+B)x+(A+2B)$$ Coefficient of like power are equated to give $A+B=1$ and $A+2B=3$ i.e $A=-1$ and $B=2$. Thus the partial fractions becomes $$\int \frac{2}{x+1}dx - \int \frac{1}{x+2}dx$$
\textbf{By Parts} \\
A method of integration using the formula $$\int u \frac{dv}{dx}dx=uv-\int v \frac{du}{dx}dx$$ For example, it is possible to integrate $xcosx$ using $x=u$ and $cosx = \frac{dv}{dx}$ so that $\frac{du}{dx}=1$
and $v=sinx.$ Then the formula gives $$\int x\cos xdx=x\sin x- \int \sin xdx$$ $$=x\sin x+\cos x$$
\textbf{Other types}\\
1. $\int \frac{dx}{ax^2+bx+c},\int \frac{dx}{\sqrt{ax^2+bx+c}},\int \sqrt{ax^2+bx+c} \medspace dx$\\ $\implies$ Put $x+\frac{b}{2a}=t$\\ \\
2. $\int \frac{px+q}{ax^2+bx+c}dx,\int \frac{px+q}{\sqrt{ax^2+bx+c}}dx$,\\$\int(px+q)\sqrt{ax^2+bx+c} \medspace dx \implies$ \\
Put $x+\frac{b}{2a}=t$ then split the integral. \\ \\
3. $\int \frac{dx}{a+b\sin^2x}, \int \frac{dx}{a+b\cos^2x},\int \frac{dx}{a\sin^2x+b\sin x \cos x+c\cos^2x}$ \\
$\implies$ Put $\tan x=t$ \\ \\
4. $\int \frac{dx}{a+b \sin x}, \int \frac{dx}{a+b \cos x},\int \frac{dx}{a+b \sin x+c\cos x}$ \\
$\implies$ Put $\tan(\frac{x}{2})=t$ \\ \\
5. $\int \frac{dx}{(ax+b)\sqrt{px+q}},\int \frac{dx}{(ax^2+bx+c)\sqrt{px+q}} \implies$ \\
Put $px+q=t^2$ \\\\
6. $\int \frac{dx}{(ax+b)\sqrt{px^2+qx+r}} \implies$ Put $ax+b=\frac{1}{t}$ \\\\
7. $\int \frac{dx}{ax^2+b\sqrt{px^2+q}} \implies$ Put $x=\frac{1}{t}$ \\\\
8. $\int \frac{x^2+1dx}{x^4+\lambda x^2+1}$ where $\lambda$ is any constant $\implies$ Divide numerator and denominator by $x^2$ and Put $x \mp \frac{1}{x}=t$
\subsection{Reduction Forms}
\begin{itemize}
\item $\int \sin^{n}xdx= \frac{-\sin^{n-1}x\cos x}{n}+ \frac{n-1}{n} \int \sin^{n-2}xdx$ 
\item $\int \cos^{n}xdx= \frac{\cos^{n-1}x \sin x}{n}+ \frac{n-1}{n} \int \cos^{n-2}xdx$
\item $\int e^{ax}\sin bxdx = \frac{e^{ax}}{a^2+b^2}(a\sin bx-b\cos bx)$
\item $\int e^{ax}\cos bxdx = \frac{e^{ax}}{a^2+b^2}(a\cos bx+b \sin bx)$
\item $\int (lnx)^{n}dx = x(lnx)^{n} -n \int (lnx)^{n-1}dx$
\item \textbf{For n$>$1} 
\item $\int \tan^{n}xdx= \frac{\tan^{n-1}x}{n-1}- \int \tan^{n-2}xdx$ 
\item $\int \sec^{n}xdx= \frac{\sec^{n-1}x\sin x}{n-1}+ \frac{n-2}{n-1} \int \sec^{n-2}xdx$ 
\end{itemize}
\subsection{Definite Integration}
\subsubsection{Properties}
\begin{itemize}
\item $\int_{a}^{b}f(x)dx=\int_{a}^{b}f(t)dt$
\item $\int_{a}^{b}f(x)dx=-\int_{b}^{a}f(x)dx$
\item $\int_{a}^{b}f(x)dx=\int_{a}^{c}f(x)dx+\int_{c}^{b}f(x)dx$
\item $\int_{-a}^{a}f(x)dx=\int_{0}^{a}(f(x)+f(-x))dx=$ $\begin{cases}
  2\int_{0}^{a}f(x)dx,f(-x)=f(x)\\
  0,f(-x)=f-(x)
\end{cases}$
\item $\int_{a}^{b}f(x)dx=\int_{a}^{b}f(a+b-x)dx$
\item $\int_{0}^{a}f(x)dx + \int_{0}^{f(a)}f^{-1}(x)dx = af(a)$
\end{itemize}

\textbf{If $f(x)$ is a periodic function with period $T$}
\begin{itemize}
\item $\int_{0}^{nT}f(x)dx=n\int_{0}^{T}f(x)dx,n \in \mathbb{Z}$
\item $\int_{a}^{a+nT}f(x)dx=n\int_{0}^{T}f(x)dx, n \in \mathbb{Z},a \in \mathbb{R}$
\item $\int_{mT}^{nT}f(x)dx=(n-m)\int_{0}^{T}f(x)dx, m,n \in \mathbb{Z}$
\item $\int_{nT}^{a+nT}f(x)dx=\int_{0}^{a}f(x)dx, n \in \mathbb{Z}, a \in \mathbb{R}$
\item $\int_{a+nT}^{b+nT}f(x)dx=\int_{0}^{a}f(x)dx,n \in \mathbb{Z}, a,b \in \mathbb{R}$
\end{itemize}
\subsubsection{Inequalities}
1. If $\Psi(x) \le f(x) \le \phi(x)$  for $a \le x \le b$,then
$$\int_{a}^{b}\Psi(x)dx \le \int_{a}^{b}f(x)dx \le \int_{a}^{b}\phi(x)dx$$
2. If $m \le f(x) \le M$ for $a \le x \le b$,then 
$$m(b-a) \le \int_{a}^{b}f(x)dx \le M(b-a)$$
3. If $f(x) \ge 0$ on $[a,b]$,then $$\int_{a}^{b}f(x)dx \ge 0$$
\subsubsection{Leibniz Theorem} 
If $\varphi(x)=\int_{g(x)}^{h(x)}f(t)dt$, then
$$\frac{d}{dx}(\varphi(x))=h'(x)f(h(x))-g'(x)f(g(x))$$
\section{Other Integrals}
\subsection{Wallis' Integral}
$$\int_{0}^{\frac{\pi}{2}} \sin^{n}xdx/\int_{0}^{\frac{\pi}{2}} \cos^{n}xdx=
\begin{cases}
  \frac{\pi}{2}\frac{(n-1)!}{n!}, \text{n is even}\ \\
  
  \frac{(n-1)!}{n!},\text{n is odd}\
\end{cases} $$
\subsection{Pi Function}
$$\Pi(n)= \int_{0}^{\infty}x^{n}e^{-x}dx$$

Properties:
\begin{itemize}
\item $\Pi(n+1)=(n+1) \Pi(n)$
\item $\Pi(0)=1 \implies \Pi(n)=n!$
\end{itemize}
\subsection{Gamma Function}
$$\Gamma(n)=\Pi(n-1)= \int_{0}^{\infty}x^{n-1}e^{-x}dx$$

Some Properties:
\begin{itemize}
\item $\Gamma(n+1)=n \Gamma(n)$
\item $\Gamma(n)=(n-1)!, \thickspace \Gamma(\frac{n}{2})=\frac{2^{(1-n)} (n-1)!\sqrt{\pi}}{(\frac{n-1}{2})!}$
\end{itemize}
\subsection{Gaussian Integral}
$$\int_{-\infty}^{\infty}e^{-x^{2}}dx=\sqrt{\pi}$$
\subsubsection{Gaussian Integral Proof}
\begin{proof}
Substitute $x^2=u \implies 2xdx=du$ \\
$$\int_{-\infty}^{\infty}e^{-x^{2}}dx \implies \frac{1}{2}\int_{-\infty}^{\infty} u^{-\frac{1}{2}}e^{-u}du$$
Using property $[\int_{-a}^{a}f(x)dx=2 \int_{0}^{a}f(x)dx]$ for $[f(-x)=f(x)]$ \\
$$=\int_{0}^{\infty} u^{-\frac{1}{2}}e^{-u}du$$
It is type of $\Gamma(n)$ for $n=\frac{1}{2}$
$$\Gamma(\frac{1}{2})=\sqrt{\pi}$$
Hence proved.
\end{proof}
\section{Differential Equation}
A relationship between between an independent variable $x$, a dependent variable $y$, and one or more of derivatives of $y$ $w.r.t$ $x$. \\
A simple example of differential equation is \\
$$\frac{dy}{dx}=x$$
\textbf{Order and Degree of DE}\\
\textbf{Order:} The order of highest-order derivative in a differential equation. \\
\textbf{Degree:} The power to which the highest-order derivative is raised in a differential equation.\\
\textbf{Solution}\\
A solution of a differential equation is function that, when substituted for the dependent variable in equation, leads to an identity. Thus for above example $y=\frac{1}{2}x^2+c$ is a $solution.$
\subsection{DE of first order and first degree}
\subsubsection{Exact Equation}
Equation of the form: $$P(\frac{dy}{dx})+	Q=0$$ are exact if left-hand side is differential coefficient of some function $f(x,y)$ $w.r.t$ $x.$. Integration gives the $solution$ $f(x,y)=C$. An $exact$ equation is one in which the total differential of function $f$ is equal to zero.
$$\frac{\partial f}{\partial x}dx+\frac{\partial f}{\partial y}dy=0$$
Thus an equation $Ax+by=0$ is exact if
$$\frac{\partial A}{\partial y}=\frac{\partial B}{\partial x}$$
\subsubsection{Variables Separable}
In this case, the equation can be written in the form $$f(x)+g(x)\frac{dy}{dx}=0$$ Rearrangement gives $$f(x)dx=-g(y)dy$$
Both sides then can be integrated.
\subsubsection{Homogeneous Equations}
These can be written in the form
$$\frac{dy}{dx}=f(\frac{y}{x})$$
The method of solution is to make substitution $y=vx$, which reduces the equation to one in $v$ and $x$ only. Resulting, the variables are separable.
\subsubsection{Equations reducible to Homogeneous}
Equation of the form  $$\frac{dy}{dx}=\frac{a_1x+b_1y+c_1}{a_2x+b_2y+c_2}$$ can be handled by making substitution $x=X+h$
and $y=Y+k$ where $h$ and $k$ are constants.Then, $$\frac{dy}{dx}=\frac{dY}{dX}$$ $$=\frac{a_1(X+h)+b_1(Y+k)+c_1}{a_2(X+h)+b_2(Y+k)+c_2}$$
If $h$ and $k$ are chosen to be the values of $x$ and $y$, respectively, that satisfy the simultaneous equations
$$a_1x+b_1y+c_1=0$$
$$a_2x+b_2y+c_2=0$$
Then original equation becomes $$\frac{dY}{dX}=\frac{a_1X+b_1Y}{a_2X+b_2Y}$$ which is homogeneous. \\
However if $\frac{a_1}{a_2}=\frac{b_1}{b_2} \not= \frac{c1}{c2}$ then $h$ and $k$ cannot be chosen as above.In this case, let $a_2=ma_1$ and $u=a_1x+b_1y$ The equation becomes $$\frac{du}{dx}-a_1=b_1 \frac{u+c_1}{mu+c_2}$$ and $u$ and $x$ can be separated.
\subsubsection{Linear Equations}
Equation of the form $$\frac{dy}{dx}+Py=Q$$ where $P$ and $Q$ are the functions of $x$, or constants, are said to be linear in $y$ and can be solved by multiplying integrating factor $$e^{\int Pdx}$$This makes left hand side of equation an exact differential:
$$e^{\int Pdx}(\frac{dy}{dx})+e^{\int Pdx}(Py)=e^{\int Pdx}Q$$
$$\frac{d}{dx}[e^{\int Pdx}y]=e^{\int Pdx}Q$$
$$ye^{\int Pdx}=\int e^{\int Pdx}Qdx+c$$
\subsection{Bernoulli's Differential Equation}
A first order differential equation of the form 
$$\frac{dy}{dx}+P(x)y=Q(x)y^n, n \in \mathbb{R}$$
\subsubsection{Transformations}
When $n=0$ the differential equation is linear and $n=1$, it is variable separable.For $n \not= 0,1$ The substitution $u=y^{1-n}$ reduces Bernoulli equation to linear differential equation.
$$\frac{du}{dx}-(n-1)P(x)u=-(n-1)Q(x)$$ 
For example: \\
In case of $n=2$, making substitution $u=y^{-1}$ in the differential equation 
$$\frac{dy}{dx}+\frac{1}{x}y=xy^2$$ produces the equation $$\frac{du}{dx}-\frac{1}{u}=-x$$ which is a linear equation.
\end{document}
